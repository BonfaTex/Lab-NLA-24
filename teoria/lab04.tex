%!TEX root = ../main.tex

\chapter{Lab 04: Iterative solvers and preconditioners with Eigen}
\thispagestyle{empty}

In this lab we aim at evaluating some hand-made implementation of the most common iterative methods for solving linear systems.

\section{Hand-made Conjugate Gradient method}

First, we create a new directory called \texttt{iter\_sol++} and we copy the following implementation of the Conjugate Gradient method in a cpp file called \texttt{eser1.cpp}:

\begin{lstlisting}
namespace LinearAlgebra
{
template <class Matrix, class Vector, class Preconditioner>
int CG(const Matrix &A, Vector &x, const Vector &b, const Preconditioner &M,
   int &max_iter, typename Vector::Scalar &tol)
{
  using Real = typename Matrix::Scalar;
  Real   resid;
  Vector p(b.size());
  Vector z(b.size());
  Vector q(b.size());
  Real   alpha, beta, rho;
  Real   rho_1(0.0);

  Real   normb = b.norm();
  Vector r = b - A * x;

  if(normb == 0.0)
    normb = 1;

  if((resid = r.norm() / normb) <= tol)
    {
      tol = resid;
      max_iter = 0;
      return 0;
    }

  for(int i = 1; i <= max_iter; i++)
    {
      z = M.solve(r);
      rho = r.dot(z);

      if(i == 1)
        p = z;
      else
        {
          beta = rho / rho_1;
          p = z + beta * p;
        }

      q = A * p;
      alpha = rho / p.dot(q);

      x += alpha * p;
      r -= alpha * q;

      if((resid = r.norm() / normb) <= tol)
        {
          tol = resid;
          max_iter = i;
          return 0;
        }

      rho_1 = rho;
    }

  tol = resid;
  return 1;
}
} // namespace LinearAlgebra   
\end{lstlisting}

